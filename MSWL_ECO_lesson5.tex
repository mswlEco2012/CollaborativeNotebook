\chapter{Libre software and open innovation}

\section{Introduction}\label{lesson-5-introduction}
FLOSS is an example of model of innovation in software and it is interesting how this model is developed by open communities, in some way similar to another business areas.  

Sustainability of a FLOSS project is the capable funding option of being maintained at a steady level, we could identify the following funding:
 	\begin{itemize}
		\item External funding
		\item Income comes from activities of the organization
		\item Developments without direct funding
		\item Developments for internal use
		\item Mixed models
 	\end{itemize}
 	
% open innovation and open source 
% done by Amal Roumi

\section{Open source and Open innovation} \label{Open source and Open innovation}
\"Open innovation is a paradigm that assumes that firms can and should use external ideas as well as internal ideas, and internal and external paths to market, as the firms look to advance their technology\"
'Open innovation' is a term coined by Professor of Business Henry Chesbrough in his 2003 book Open Innovation: The New Imperative for Creating and Profiting from Technology. 

In the years since its publication, Chesbrough\'s ideas on how technology should be managed and exploited have become extremely influential. Over the same period, the public profile of free and open source software (FOSS) has risen. 

So we can say that FOSS is an example of open innovation in software.
They have two main element: use the technology, and collaborative development of that technology, unlike many individual participants, companies must also consider an economic return to justify there investment in.


\subsection{Patterns of open innovation in free software} \label{Patterns of open innovation in free software}
There are four patterns of open innovation in free software:
\begin{itemize}
 \item  \textbf {Pooled RnD}: Pool resources to innovate in a common platform, exploit results.
 \begin{itemize}
		\item Maximization: concentrate in their own needs.
		\item Incorporation: shared technology in their products.
		\item Motivation: pool of contributors assumed.
	\end{itemize}
Lunix, Mozilla For both donate RnD to open source project while exploiting the pooles RnD of all contributors to facilitate the sale of related products.


 \item  \textbf {Spintouts}: Release the poetical of technologies within the firm that are not creating value.and they have opportunities to release more value from their technologies by position them out the firm but at the same time maintaing an ongoing corporate involvement. Here the \"spinout\"come.
	\begin{itemize}
		\item Maximization of impact of non-core technologies.
		\item Incorporation of contributions by third parties.
		\item Motivation: self-sustainable (or less resource-consuming) communities.
	\end{itemize}
Eclipse, Beehive,Jikes projects the sponsor firms spinout open source project that were closely aligned with firm\'s ongoing strategies

 \item \textbf{Selling complements:} Income from complements, shared innovation in a common core.
	\begin{itemize}
		\item Maximization by centering on core products.
		\item Incorporation of "free" external innovation.
		\item Motivation: self-sustainable (or less resource-consuming) communities.
	\end{itemize}
Two open source examples are the IBM's WebShere and Apple's Safari browser.
Android, KDE, Apache, Konqueror, Darwin open source project the firm adopting open source components.

 \item \textbf{Donated complements:} firms make their money off of the core innovation, but seek donated labor for valuable complements. Firms have indirectly and directly supported users collaboration that is coordinated using open source technologies.

	\begin{itemize}
		\item Maximization: more value for internal innovation (coreproduct).
		\item Incorporation: complements are attracted innovation.
		\item Motivation: developers involved in the core product, but willing more functionality.
	\end{itemize}
Examples: Early BSD Unix, Matlab Central, PC Game "Mods",Avalache .
\end{itemize}

 %  Amal Roumi
 
 \section{Sustainability}\label{Sustainability}
The FLOSS communities and their projects are like living being, therefore they need food to stay alive. That is, will the community have the enough resources to be sustainable?, to answer this question is really important for the future of a FLOSS project, and how communities find funds by companies or organizations create different models of sustainability. 
  
 	\subsection{Funding options for libre software}\label{Funding options for libre software}
 	Initially this classification was treated for business model.
 
		\subsubsection{External funding}
		
		Public administrations usually have the following motivations:
			\begin{itemize}		
				\item \emph{Scientific motivation}: that software required to produce certain results must be available, in order to reproduce them.
				\item \emph{Precompetitive motivation}: all industry network can benefit from precompetitive results. 
				\item \emph{Promoting standards motivation}: Implementing reference versions of a certain standard.
				
				\item \emph{Social motivation}: Funding creation of common basic infrastructure of information society. 
				
				\item \emph{Case study}: Gnat (Ada compiler), \$1M assigned from U.S. Government to NYU for its development. And the curious was the development of this product was initially valued around \$100M.
			\end{itemize}
							
			For instance, the German Government has issued a DM 250,000 grant to the GNU Privacy Guard (GPG)~\cite{GPG} project. The grant will help the GPG team develop for more platforms and develop easy to use interfaces and APIs. Another examples of funding by public administrations are the Spanish States Government,such as Extremadura, Andalucia, Valencia and others.
			
			Others external fundings: 			
			
			\begin{itemize}	
			\item \emph{Non-profit organization}: This foundation usually made for non-profit orgarnization. The funding mechanisms can be direct or indirect.  For example, \emph{MediaWiki} is a free wiki software application, developed by the Wikimedia Foundation and others, it is used to run all of the projects hosted by the Foundation, including Wikipedia, Wiktionary and Commons. \emph{Blender} is a free and open-source 3D computer graphics software product used for creating animated films, visual effects, interactive 3D applications or video games; Blender was developed as an in-house application by the Dutch animation studio Neo Geo and Not a Number Technologies (NaN). The \emph{Open Bioinformatics Foundation} is a non profit, volunteer run organization focused on supporting open source programming in bioinformatics. 
			\item \emph{Indirect}: Some companies funds FLOSS projects to achive profits with something associated with this project. Therefore Google funds Android project because it wants to achieve mobile phone technology positioning. Another example is O'Reilly that funds FLOSS projects to sell books about them. Many hardware companies funds FLOSS projects to improve their firewires and drivers. Red Hat, a company of services, funds Gnome projects because its core product uses it.
			
			\item \emph{Need of improvements}: Some companies funds FLOSS project to improve requirements of their products, saving resources in contract with the traditional development process. So for example Corel funds Macadamian to Wine to achieve a Corel Draw working in linux.
			
			\end{itemize}
		
		\subsubsection{Income comes from activities of the organization(s)}
			\begin{itemize}	
				\item \emph{Detailed knowledge}: Corporations fund in some leading project, usually for having detailed knowledge about it. For example, many corporations are collaborating in WebKit project to achieve in mobile phone market, besides small ones like Igalia can achieve this technology positioning, and in with detailed knowledge they can sell its services.

Autofinanciado: venta de marca
%-----------------------------------
				\item \emph{Brand selling}: If a brand is sufficiently known, it is easier to sell services around it. For examble, Mozilla with Firefox is using this model. Even more clear, it is Red Hat with its core product (Enterprise Red Hat OS), and so brand "Red Hat" means detailed knowledge of linux for customers.

			\end{itemize}
		\subsubsection{Developments without direct funding} 
		Most of the projects of FLOSS are development in this way. In many cases, there are indirect funding: Some companies dedicate human resources to develop the project for different reasons, some organization fund this project to improve some feature interested for them, donation of money or IT infrastructure.
		For example, initially OpenStack is here, after it had a relative success, a foundation was created with \$10M to promote this project.
		
		\subsubsection{Developments for internal use}
		For instance, Cisco used this way of funding for development of tools designed to optimize the management of large numbers of printers (CEPS, Cisco Enterprise Print System ~\cite{CEPS}).
		
		\subsubsection{Mixed models}
		The reality is that the models are mixed and project usually move among them along its history.

\section{Conclusions}\label{conclusions}
Open innovation is "new" paradigm, which is working properly in FLOSS project, but it is more important to understand that it is a model that can work perfect many areas.
Sustainability is essecial feature for FLOSS project viability, there are many way for fundings, and projects and organizations usually use mixed models and change along the life of the project.
