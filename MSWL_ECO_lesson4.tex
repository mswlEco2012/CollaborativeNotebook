\chapter{FLOSS business models}

\section{Introduction}\label{lesson-4-introduction}

FLOSS 

The one million dollar question \emph{Which model has success.}

Different FLOSS business models inside the market. Those sample are from two years ago, so don't take any strict conclusions because this models are continusly evolving.

\subsection{Origin}\label{lesson-4-origin}

Open innovation rise business models.

\subparagraph{Main Content}

Assets of the company to make money, culture, employees and knowledge as a part
of the capital that we will put aside money.

Open innovation stares at knowlegment and its use.

We begin with the example of a company where its product is the reference and
not the knowledge of its employees to manufacture the product, this case would
be:

If only produces the product strictly like a dictate of the company , although
exists an employee who knows how to manufacture the product more better than the
company its self therefore is underused, the resource is maximized.

Unlike the knowledge unification among employees so as to find the best way for
developing the product in which they all participate and the best improvments or
suggestions tending to envolve are welcomed as nothing is permanent.

How companies manage the knowledge ? \emph{The knowledge is essential for the
company and therefore it has to protect}, this is against with the
principles of  FLOSS, so as the competion does not compete with their
knowledge. The private treatment of the knowledge restricts the radius of
improvement over the different cases in which the knowledge is being shared because by sharing knowledge they both
win. The rapid innovation against the protection of the company - of the
enviroment where closes to outside opinions.

The knowledge exchange begins between employees by companies
as for example in case of Silicon Valley at the middle of 70's where the
companies needed engineers and an engineer's life in a company was no more than
18 months so the knowledge was flowing between all companies until convincing a critical mass
knowledge exchange for the transfer of employees.

\begin{itemize}
  \item Case Study: \emph{Gold Corp} Predictive model.
  \item Case Study: \emph{Netflix}
\end{itemize}

Predictive models developed by people outside of the company rewarding the ``winning model''

Beginning with NDAs (non-disclosure agreement) as a point in Software
companies,they delimit communication and dissemination of knowledge. 

Until the gap came in where people believed that innovation should not be the
solely subject to the company and that in this way it would grow faster, the
feedback and the contact with 'competition'.

This model could be extrapolated to any other business, for example the
inclusion of customer in the design process to receive feedback from the user community.

Kickstarter:Leaving the market foresight to the customers before the prouduct is
being manufactured

\section{Current trends}\label{lesson-4-current-trends}

\emph{"There is no silver bullet"}

Be attentive to the market , not only to which one belongs but to external
markets so that can be defined as a specialization or an expertizing of the
market.It rewards collaboration with users generating knowledge on par with innovation
so that the interested user could improve the product.

\section{False myths}\label{lesson-4-false-myths}

The sales price is the price which people are able to pay , and it has nothing
to do with the production cost. 
The value that is given by user/client can be evaluated : 


\begin{itemize}
    \item The value of using the tool , investment.
    Savings of 5000 euros  for one computer but you pay 500 Euros as a sale
    value.
    \item The sales value.
\end{itemize}


The 75\% of the programs is dedicated to the maintenance while the rest is
dedicated to the innovation.
The value of software is not measured by the \emph{value of replacement}. 

It doesn't mind the cost of the production but what it would cost if you don't
having to produce the product again, associated with the physical assets.


Expected value of the future service.

The upper limit in the software is marked by future service that offers to me ,
by looking forward in what he has not been done so far.

Software stability , support, documentation , adaption or how can you get a
certification.Unlike the physical goods you don't think as soon as you can find
the same product as they can get the same product updated while that in a
physical if you look back.

    Services == Source of revenues

In Software you don't purchase the product , you evaluate the service that product offers.

The organization of Free Software model allows offering best services:
\begin{itemize}
    \item Users who become clients.
    \item Scalability in failure identification. Successful programs use to have less errors, where success measured in user.
    \item Share risks and production costs. Development cost is shared among the actors.
    \item Monopolistic practices are very difficult. A fork allows to escape from product monopolization.
\end{itemize}

\emph{If you don't have users , you don't have any available business model.}
%\end{Main Content}

\section{A guide for SMEs}

FLOSS by itself is not, and it has never been, a business model.

\subsection{Where is FLOSS ?}\label{sec:floss-bm}

Mobiles: sell mobiles as a business model using FLOSS, more viability, cost
 structure, not the model itself.

If the company

Si la empresa no utilizara FLOSS no sería una empresa viable, es decir, sin el
 uso de FLOSS la empresa no existiría.

\subsection{FLOSSMETRICS}

Study from 2008/2009 analysing 218 companies that 25\% of their total revenues
 directly or indirectly from FLOSS\footnote{http://libresoft.es/research/projects/flossmetrics}.

FLOSSMetrics stands for Free/Libre Open Source Software Metrics. 
The main objective of FLOSSMETRICS was to construct, publish and analyse a large
 scale database with information and metrics about libre software development 
 coming from several thousands of software projects, using existing 
 methodologies, and tools already developed.

\subsubsection{3 axes}

\begin{itemize}
    \item Software model
    \begin{itemize}
        \item Propietary vs libre software
    \end{itemize}
    
    \item Development model
    \begin{itemize}
        \item Barries to collaboration
        \item Single developer/reducted group vs. large communty, global outrech.
    \end{itemize}
    
    \item Business model
    \begin{itemize}
        \item Type of revenues model.
        \item Numerous options: Training, support, on-demand changes, productizing, SaaS, etc.
    \end{itemize}
\end{itemize}

Development model samples:
\begin{itemize}
  \item Google Android - Private production, central model.
  \item GCC - Libre, low barriers, medium group.
  \item GLibC - Libre, low barriers, little group.
\end{itemize}

\section{Strategic uses of Libre Software}

\emph{You do not change the traditional business model, but you lean on FLOSS
 products to enhance the model.}

It is very difficult to find a company that does not use any of FLOSS. 
FLOSS is adaptive to the requirements of a company. 
Translations, adaptations to different types of hardware, interoperability, 
organizational environments, culture, new technologies.

Otra tendencia, es la estrategia de liberar distintas distribuciones de 
software como FLOSS para crear una comunidad libre alrededor.

Apple, Microsoft, Facebook, release parts of your code to create a community to 
keep the product by the same community (composed of these same companies also).

\subsection{Imbalances}

Users who do not want to pay for proprietary software and are satisfied with 
the program free of charge.

They can improve the proprietary model. The proprietary is challenged when a 
FLOSS reliable solution generates an acceptance in the market. (sitemas embedded)

\section{Carlo Daffara taxonomy}

\begin{itemize}
    \item Dual licencing: FLOSS version and propietary version. 
    MySQL Enterprise and MySQL FLOSS.
    \item Open core: Allows mixing FLOSS and proprietary elements. 
    MSExcange mail model.
    \item Product specialists: Superior knowledge, additional services. 
    Best service for the product.
    \item Platform providers: Integration, product testing. 
    PaaS: FLOSS producto integration, glue products. Test by your own.
    \item Aggregate support providers: 
    First level of support for different types of free software. 
    Scale the support from all levels.
    \item Selection/consulting companies: Closer to the analyst role, 
    minimum impact on FLOSS communities.
    \item Legal certification and consulting : Assessment on license 
    compatibility. Certificate interoperability.
    \item Training and documentation: Either as part of a broader support 
    contract or companies exclusively devoted to this market area. U
    ser guides, documentation, courses.
    \item R\&D cost sharing : Initial investment + creating community to 
    reduce R\&D costs. Create a community to reduce I+D cost
    Making people contribute from outside to keep up with the community tool.
    OpenStack from NASA throught RackSpace sharing development cost.
    \item Indirect revenues: Baseline for sales of associated products or 
    services (commodities).
\end{itemize}

\subsection{Dual (or multiple) licensing)}

You must be the owner of all code. Contributors transfer their copyright to
license the product. The simple fact of having to sign a document pulls back 
the community contributions.

Example of MySQL.
Integrated with other products to the product not being liberate private and 
thus not violating the GPL.
Related to the license, the Apache License and proprietary, had not worked 
since Apache allows the use of their code in a private product.

\subsection{Open core}

Revenues or profits focus on proprietary components around
Open Core. Being the same company, the only one that can develop
improvement of its services. The entire core is free.

Weakness: Develop proprietary service by FLOSS, replicating their
functionalities, reimplementation of a popular product.

They usually have dual licenses to provide private and FLOSS version.

Large user base, visibility and opportunity to convert to premium users to get 
the benefits.

Win brand value to be recognized by the release of the core.

Example: Implementing SSL Apache. 
Value added to the software to be the company's proprietary core.

It's a business model that generates free controversial and private components
 through its open platform, the core.

There is a doubt compenente around these projects, as they are not open to the
  whole community, to develop proprietary resources for the company.
 
A more positive view is that attracts different companies for
invest in the product to be able to develop private modules. The company spends
  a great effort on the viability of the core because if not extended their
  services would not have a business niche.

Do you earn more resources? or may be impaired by this company
behavior of the business model. What determines which one suits you
companies adopt this model?

It is very close to the proprietary model, and somehow try to keep the customer 
fenced and locked. This model may tend to disappear because it may be the case 
that another community project or company to advance faster than the solution 
offered by the company and therefore is forced to release the code, being
 too late.

The pattern moves across a time window, ie not seem feasible OpenCore lifetime 
tends to complete release code. If the product is successful, 
it is easier to occur.

Samples: SendMail, SugarCRM.

\subsection{Product specialists}

\emph{I have the best knowledge of the certified product.}

LPI y RedHat: two certifications of Linux knowledge. 

Redhat model in the field of computing Could: Include developers in
OpenStack group and become a specialist. No need to be the creator of the 
product.

Weakness: another company with better image, more name, best developments, 
overflowing your brand achieved and therefore, 
you become relegate to the background.
 
\subsection{Platform providers}

\begin{itemize}
  \item Selection, support, integration and services around a set of projects 
  integrated in a single, tested and verified product.
  \item Key points; 
  \begin{itemize}
    \item Verification services
    \item Additional services    
  \end{itemize}
  \item Copyright ownership prevents direct copy (not cloning).
  \item Example: Redhat.
  \item The source code is libre software, but the product name and logo 
  are trademarks.
  \item Considerable effort to eliminate them from files, doocumentation, etc.
\end{itemize}

Integrated business model: private label tablets, software, hardware, etc.\ldots
Multimedia players from 'Carrefour'.

Specialization and white label conversion technology.

\subsection{Aggregate support providers}

Similar to classic support but with the difference that is at all levels.

Two paths to choose from:
\begin{itemize}
  \item By using its own resources or outsourcing to solve problems.
  \item Frontdesk; handles requests by companies who know the product.
\end{itemize}

Clear benefit for large projects whose costs could raise due to excessive
 diversification of support channels (comprehensive help-desl).

OpenLogic: experts in Linux.

\subsection{Selection/consulting companies}

No software developers. They are responsible for advising the client to 
provide solutions. Testing, performance analysis, knowledge of the territory.

People expert in a field dedicated to doing analysis of a product.

These companies provide tools that help to interpret results by playing 
more complete analysis.

Open WebApps

\subsection{Legal certification/consulting}

Specializing in software licensing and code analysis for possible 
incompatibilities.

Palamida, Blackduck (mediante ohloh), Sonatype.

\subsection{Training and documentation}

Training and documentation not certifiation.

GBDirect. 

\subsection{R\&D Const sharing}

\emph{Memo (nokia)}, nokia develop only a portion and the rest would be up to the
 community. This model is before Android. They knew the cost of software but 
 were not a production company, engaged in the manufacture of terminals so 
 opted for this model, let the software grew hand in the community. KDE, QT, 
 Linux were parties that formed the core of Memo.
 
\emph{OpenStack}: Rackspace is dedicated to the cloud market, sold clouds. 
RackSpace produced 'OpenStack' sharing the cost of developing the community 
(HP, Redhat, Intel\ldots). Not your niche market, so the development of free
 software is shared with others to take advantage of the product.
 
\subsection{Indirect revenues}

Intel, Dell, O'Reilly. Perl was the first case of financing a free software
 project to Larry Wall. For Larry Wall wrote a book for his publisher (O'Reilly)
  paying the cost of the project at its own expense to get 
  the next version of Perl.


Open source going mainstream - Gartner group 
report\footnote{http://www.gartner.com/newsroom/id/593207}

\emph{By 2012, 80 per cent of all commercial software will include elements of
 open-source technology. Many open-source technologies are mature, stable and
  well supported. They provide significant opportunities for vendors and users
   to lower their total cost of ownership and increase returns on investment. 
   Ignoring this will put companies at a serious competitive disadvantage. 
   Embedded open source strategies will become the minimal level of investment
    that most large software vendors will find necessary to maintain
     competitive advantages during the next five years.}

\section{Conclusions}\label{conclusions}
