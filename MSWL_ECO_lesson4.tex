\chapter{FLOSS business models}

\section{Introduction}\label{lesson-4-introduction}

FLOSS 

The one million dollar question \emph{Which model has success.}

Different FLOSS business models inside the market. Those sample are from two years ago, so don't take any strict conclusions because this models are continusly evolving.

\subsection{Origin}\label{lesson-4-origin}

Open innovation rise business models.

% I've commented this section to avoid compiling problems. commet is defined in
% verbarim package, use it as a normal item after importing in main file.

\begin{comment}

Activos de la empresa para ganar dinero, cultura, empleados y conocimiento a parte del capital monetario que dejaremos de lado. Open innovation stares at knowlegment and its use.

Empezaremos con el ejemplo de una empresa donde su producto es el referente y no el conocimiento de los empleados para fabricar el producto, el caso sería el siguiente:

Si sólo se produce el producto estrictamente como dicta la empresa aunque exista un empleado que sepa construir el producto mucho mejor de lo que está haciendo la empresa por lo tanto está infrautilizado, el recurso se aprovecha al máximo. Al contrario que unificar el conocimiento entre los empleados para consesuar una mejor forma de construir el producto en la que participan todos y las mejoras o sugerencias son escuchadas tendiendo a evolucionar ya que nada es permanente.

¿ Como gestionan las empresas el conocimiento ? \emph{El conocimiento es básico para la empresa y por ello ha de protegerlo}, esto rompe con los pricipios del FLOSS, para que la competencia no compita con su propio conocimiento.

El tratamiento privativo del conocimiento restringe el radio de mejora frente a los diferentes casos comunes en los que se comparte el conocimiento, ya que de esta forma al compartir el conocimiento ambos ganan. La innovación rápida frente a la protección de la empresa del entorno en donde se cierra a opiniones externas.

El intercamio de conocimientos comienza entre el intercambio de empleados por parte de las empresas, como ejemplo el caso de Silicon Valley a mediados de la década de los 70 en donde las empresas necesitaban ingenieros y la vida de un ingeniero en una empresa era de 18 meses por lo que fluía el conocimiento entre todas las empresas hasta concevir una masas crítica a partir del intercambio de conocimientos por el trasvase de empleados.

\begin{itemize}
  \item Case Study: \emph{Gold Corp} Modelo predictivo.
  \item Case Study: \emph{Netflix}
\end{itemize}

Modelos predictivos desarrollados por personas externas a la misma empresa premiando al modelo ganador.

Empezando por los NDAs (non-disclosure agreement) como puntal en las empresas de Software, acotan la comunicación y la difusión del conocimiento. Hasta que vino la brecha en donde la gente creyó en que la innovación no debería estar sujeta únicamente a la empresa y que de esta manera crecería más rápidamente el mismo, la realimentación y el contacto con 'la competencia'. Este modelo es extrapolable a cualquier otro negocio, como ejemplo la inclusión de los clientes en el proceso de diseño recibiendo un feedback de la comunidad de usuarios.

Kickstarter: dejando en manos de los clients la prospectiva de mercado previamente a que el producto se fabrique.

\section{Current trends}\label{lesson-4-current-trends}

\emph{"There is no silver bullet"}

Estar atento al mercado, no sólo al que se pertenece si no a los mercados externos por lo que se podría definir como una especialización del mercado o expertización. Se premia la colaboración con los usuarios generando conocimiento a la par que la innovación, por lo que el usuario que tenga interés puede mejorar el producto.

\section{False myths}\label{lesson-4-false-myths}

El precio de venta es el que está dispuesto a pagar la gente, no tiene nada que ver con el coste de producción.

El valor que le da el usuario/cliente puede evaluarse:

\begin{itemize}
    \item El valor del uso de la herramienta, inversión. Ahorro de 5000 euros por un ordenador pero se paga 500 euros por el como valor de venta.
    \item El valor de venta.
\end{itemize}

El 75\% de los programas se dedica al mantenimiento mientras que el resto se dedica a innovación.

El valor del software no se mide a partir del \emph{valor de reemplazo}. No importan los costes de producción si no de cuánto costaría tener que producir el producto de nuevo, asociado a los bienes físicos.

Expected value of the future service.

EL límite alto en el software está marcado por el servicio futuro que me ofrece, mirando hacia adelante no lo que ha hecho hasta ese momento. Estabilidad del software, soporte, documentación, adaptación o como se puede conseguir una certificación. A diferencia de los bienes físicos no se piensa en cuanto se puede conseguir el mismo producto actualizado mientras que en un físico si que se mira atrás.

    Services == Source of revenues

En Software no compras el producto, valoras el servicio que te ofrece.

La organizaciónd el modelo de Software Libre permite ofrecer mejores servicios:
\begin{itemize}
    \item Usuarios que se convierten en clientes.
    \item escalabilidad en identificación de fallos. Los programas de éxito suelen tener menos errores, el éxito medido en usuario.
    \item Compartir riesgos y costes de producción. Entre los actores se comparte el gasto del desarrollo.
    \item Prácticas monopolísticas muy difíciles. Un fork permite escapar de la monopolización del producto.
\end{itemize}

\emph{Si no tienes usuarios no tienes modelo de negocio.}
\end{comment}

\section{Part II}\label{Part II} % TODO: Change for properly section title.

...

\section{Conclusions}\label{conclusions}
