\chapter{FLOSS business models}

\section{Introduction}\label{lesson-4-introduction}

FLOSS 

The one million dollar question \emph{Which model has success.}

Different FLOSS business models inside the market. Those sample are from two years ago, so don't take any strict conclusions because this models are continusly evolving.

\subsection{Origin}\label{lesson-4-origin}

Open innovation rise business models.

 I've commented this section to avoid compiling problems. commet is defined in
 verbarim package, use it as a normal item after importing in main file.

%\begin{comment}
\subparagraph{Main Content}

Assets of the company to make money, culture, employees and knowledge as a part
of the capital that we will put aside money.

Open innovation stares at knowlegment and its use.

We begin with the example of a company where its product is the reference and
not the knowledge of its employees to manufacture the product, this case would
be:

If only produces the product strictly like a dictate of the company , although
exists an employee who knows how to manufacture the product more better than the
company its self therefore is underused, the resource is maximized.


Unlike the knowledge unification among employees so as to find the best way for
developing the product in which they all participate and the best improvments or
suggestions tending to envolve are welcomed as nothing is permanent.

How companies manage the knowledge ? \emph{The knowledge is essential for the
company and therefore it has to protect}, this is against with the
principles of  FLOSS, so as the competion does not compete with their
knowledge. The private treatment of the knowledge restricts the radius of
improvement over the different cases in which the knowledge is being shared because by sharing knowledge they both
win. The rapid innovation against the protection of the company - of the
enviroment where closes to outside opinions.

The knowledge exchange begins between employees by companies
as for example in case of Silicon Valley at the middle of 70's where the
companies needed engineers and an engineer's life in a company was no more than
18 months so the knowledge was flowing between all companies until convincing a critical mass
knowledge exchange for the transfer of employees.

\begin{itemize}
  \item Case Study: \emph{Gold Corp} Predictive model.
  \item Case Study: \emph{Netflix}
\end{itemize}

Predictive models developed by people outside of the company rewarding the ``winning model''

Beginning with NDAs (non-disclosure agreement) as a point in Software
companies,they delimit communication and dissemination of knowledge. 

Until the gap came in where people believed that innovation should not be the
solely subject to the company and that in this way it would grow faster, the
feedback and the contact with 'competition'.

This model could be extrapolated to any other business, for example the
inclusion of customer in the design process to receive feedback from the user community.

Kickstarter:Leaving the market foresight to the customers before the prouduct is
being manufactured

\section{Current trends}\label{lesson-4-current-trends}

\emph{"There is no silver bullet"}

Be attentive to the market , not only to which one belongs but to external
markets so that can be defined as a specialization or an expertizing of the
market.It rewards collaboration with users generating knowledge on par with innovation
so that the interested user could improve the product.

\section{False myths}\label{lesson-4-false-myths}

The sales price is the price which people are able to pay , and it has nothing
to do with the production cost. 
The value that is given by user/client can be evaluated : 


\begin{itemize}
    \item The value of using the tool , investment.
    Savings of 5000 euros  for one computer but you pay 500 Euros as a sale
    value.
    \item The sales value.
\end{itemize}


The 75\% of the programs is dedicated to the maintenance while the rest is
dedicated to the innovation.
The value of software is not measured by the \emph{value of replacement}. 

It doesn't mind the cost of the production but what it would cost if you don't
having to produce the product again, associated with the physical assets.


Expected value of the future service.

The upper limit in the software is marked by future service that offers to me ,
by looking forward in what he has not been done so far.

Software stability , support, documentation , adaption or how can you get a
certification.Unlike the physical goods you don't think as soon as you can find
the same product as they can get the same product updated while that in a
physical if you look back.

    Services == Source of revenues

In Software you don't purchase the product , you evaluate the service that product offers.

The organization of Free Software model allows offering best services:
\begin{itemize}
    \item Users who become clients.
    \item Scalability in failure identification. Successful programs use to have less errors, where success measured in user.
    \item Share risks and production costs. Development cost is shared among the actors.
    \item Monopolistic practices are very difficult. A fork allows to escape from product monopolization.
\end{itemize}

\emph{If you don't have users , you don't have any available business model.}
%\end{Main Content}

\section{Part II}\label{Part II} % TODO: Change for properly section title.

...

\section{Conclusions}\label{conclusions}
